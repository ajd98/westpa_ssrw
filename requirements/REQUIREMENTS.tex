\title{Requirements for w\_steadystate\_reweight.py}
\author{
        Alex DeGrave \\
       }
\date{\today}

\documentclass[10pt]{article}
\usepackage{fancyvrb}
\usepackage[margin=0.5in]{geometry}

\begin{document}
\maketitle

\begin{Verbatim}[commandchars=\\\{\},codes={\catcode`$=3\catcode`^=7\catcode`_=8}]
- Overall scheme:
  - For each input simulation, load flux matrix.
    - "flux matrix": A square, nonnegative, N*N matrix $F$ where each element
                     represents probability flow between two "bins". In this 
                     context, each element $F_{i,j}$ (where "i" is the column 
                     index, and "j" is the row index) represents flux calculated
                     from bin i, into bin j.
                     >>>DOUBLE CHECK THIS

    - Load flux matrix from output of w\_postanalysis\_matrix.py
      - Flux matrices are stored for each iteration in 
          flux\_matrix\_h5['iterations/iter\_\%08d/']
        in a sparse matrix format.  Output files from w\_postanalysis\_matrix
        store data in a manner based on the coordinate matrix format from the 
        Scipy library.  Each iteration group contains four data sets:
        1) cols:  a data set storing a vector of column indices
        2) rows:  a data set storing a vector of row indices
        3) flux:  a data set storing a vector of flux values
        4) obs:   a data set storing a vector of transition counts
        These data sets may be combined to form either the flux matrix
        (described above), or a "count matrix"
        - Count matrix: A square, integer-valued, nonnegative, N*N matrix $C$ 
                        where each element represents the COUNT of transition 
                        events between two bins.  In this context, each element
                        $C_{i,j}$ represents the number of transitions oberved 
                        from bin i, into bin j, during a given iteration.
        For example, given some integer index $k$, with $0 \le k<len(cols)$, we
        have:
             flux\_matrix[rows[k], cols[k]] = flux[k]
             count\_matrix[rows[k], cols[k]] = obs[k]

  - For each input, build a transition matrix (one per input).
    - "transition matrix": For this purpose, a transition matrix is a square,
                           nonnegative, N*N matrix $T$ where each row sums to 
                           one. More formally, it is a RIGHT STOCHASTIC matrix.  
                           Here, each element $T_{i,j}$ represents the 
                           probability that a trajectory walker in bin $i$ will
                           next transition to bin $j$ during the lag time 
                           $\tau$. 
                           >>>WHAT IS THE DEFAULT LAG TIME?

>>>>>>>>>>>>>>>>>>>>>>>>>> EDIT BELOW.  THIS IS WRONG. <<<<<<<<<<<<<<<<<<<<<<<<<
    - In the case of a simulation without recycling conditions, a transition
      matrix may be constructed from a flux matrix by row-normalizing. Such a
      transition matrix would depend only upon the events observed within a 
      single iteration.  The user may desire that multiple flux matrices are 
      represented in a single transition matrix.  Such average should tend to
      stabilize the estimate of the transition probabilites, as observations 
      from a single iteration may be insufficient to obtain reasonable
      statistics.  One may imagine multiple methods for averaging.
      1) Average the count matrices.  Given a collection of N*N count matrices
              $\{C^1, C^2, ..., C^n\}$
         the following is an estimator for the flux per iteration:
              $<C> = \frac{\Sigma_{k=1}^{n} C^k}{n}$
         The transition matrix may then be esimated by 
              $T_{i,j} = \frac{<C>_{i,j}}{\Sigma_{m=1}^{N} <C>_{i,m}}$
      2) Average the transition matrices.  Given a collection of N*N 
         count matrices 
              $\{C1, C2, ..., Cn\}$
         we first row-normalize each count matrix as 
              $T^k_{i,j} = \frac{C^k_{i,j}}{\Sigma_{m=1}^{N} C^k_{i,m}}$
         for k=1,2,...,n.  We may then obtain an estimate of the transition
         matrix $<T>$ as:
              $<T> = \Sigma_{k=1}^{n} T^k$
       Either formulation may be valid.  Method (1) places more "emphasis" on 
       high-weight transitions, in that large flux values contribute more to 
       the estimate <T> than do low-weight transitions. Method (2) places equal
       emphasis on each transition observation, regardless of the associated
       weight.

\end{Verbatim}
\end{document}

% Header, document info, and imports
\title{Requirements for w\_steadystate\_reweight.py}
\author{
        Alex DeGrave \\
       }
\date{\today}

\documentclass[10pt]{article}
\usepackage{fancyvrb}
\usepackage[margin=0.5in]{geometry}
\usepackage{amsfonts}
\usepackage{amsmath}

% Start the main part of the document
\begin{document}
\maketitle

% Verbatim preserves spacing and gives Courier font.
\begin{Verbatim}[commandchars=\\\{\},codes={\catcode`$=3\catcode`^=7\catcode`_=8}]
- Overall scheme:
  - Parse user inputs
    - Parse command line flags
      - get location of YAML file specifying input locations
      - find which iterations to include in the analysis (first and last 
        iterations)
      - find whether to calculate a single value or an evolution of values
      - get the averaging scheme to use, if any (blocked averaging, cumulative
        averaging) 
    - Parse YAML file
      for each simulation, load the:
      1) file path to the w\_postanalysis\_matrix output
      2) file path to the w\_assign output
      3) bins (or states) that were subject to recycling during the course of 
         the simulation.
        - Load keys as "bins", or "states", which should give lists as values
        - Ultimately, we need a list of bin indices.  If the user specifies
          state indices, we need to get the corresponding bin indices from the
          w\_assign output files.
          >>> where in the w\_assign output is this located? 'state\_map'? 
  - For each input simulation, load flux matrix.
    - "flux matrix": A square, nonnegative, N*N matrix $F$ where each element
                     represents probability flow between two "bins". In this 
                     context, each element $F_{i,j}$ (where $i$ is the column 
                     index, and $j$ is the row index) represents flux calculated
                     from bin $i$, into bin $j$.
                     >>>DOUBLE CHECK THIS

    - Load flux matrix from output of w\_postanalysis\_matrix.py
      - Flux matrices are stored for each iteration in 
          flux\_matrix\_h5['iterations/iter\_\%08d/']
        in a sparse matrix format.  Output files from w\_postanalysis\_matrix
        store data in a manner based on the coordinate matrix format from the 
        Scipy library.  Each iteration group contains four data sets:
        1) cols:  a data set storing a vector of column indices
        2) rows:  a data set storing a vector of row indices
        3) flux:  a data set storing a vector of flux values
        4) obs:   a data set storing a vector of transition counts
        These data sets may be combined to form either the flux matrix
        (described above), or a "count matrix"
        - Count matrix: A square, integer-valued, nonnegative, N*N matrix $C$ 
                        where each element represents the COUNT of transition 
                        events between two bins.  In this context, each element
                        $C_{i,j}$ represents the number of transitions oberved 
                        from bin $i$, into bin $j$, during a given iteration.
        For example, given some integer index $k$, with $0 \le k<len(cols)$, we
        have:
             flux\_matrix[rows[k], cols[k]] = flux[k]
             count\_matrix[rows[k], cols[k]] = obs[k]

  - For each input, build a transition matrix 
    - "transition matrix": For this purpose, a transition matrix is a square,
                           nonnegative, N*N matrix $T$ where each row sums to 
                           one. More formally, it is a RIGHT STOCHASTIC matrix.  
                           Here, each element $T_{i,j}$ represents the 
                           probability that a trajectory walker in bin $i$ will
                           next transition to bin $j$ during the lag time 
                           $\tau$. 
                           >>>WHAT IS THE DEFAULT LAG TIME?

    - In the case of a simulation without recycling conditions, a transition
      matrix may be constructed from a flux matrix by row-normalizing. Such a
      transition matrix would depend only upon the events observed within a 
      single iteration.  The user may desire that multiple flux matrices are 
      represented in a single transition matrix.  Such average should tend to
      stabilize the estimate of the transition probabilites, as observations 
      from a single iteration may be insufficient to obtain reasonable
      statistics.  One may imagine multiple methods for averaging.
      1) Average the flux matrices.  Given a collection of $N*N$ flux matrices
              $\{F^{(1)}, F^{(2)}, ..., F^{(n)}\}$
         the following is an estimator for the flux per $\tau$:
              >>>> WHAT TAU DOES w\_postanalysis\_matrix OUTPUT?
\end{Verbatim}

              \[ <F> = \frac{\sum_{k=1}^{n} F^{(k)}}{n} \]

\begin{Verbatim}[commandchars=\\\{\},codes={\catcode`$=3\catcode`^=7\catcode`_=8}]
         NOTE: It probably make more sense to restrict this sum for each row to
         only those simulations for which the bin in which a given transition
         would originate is occupied, ie:
\end{Verbatim}

              \[ <F_{i,j}> = \frac{\sum_{k\in D_{i}} F^{(k)}_{i,j}}{D_{i}^\#} \]
        
\begin{Verbatim}[commandchars=\\\{\},codes={\catcode`$=3\catcode`^=7\catcode`_=8}]
         where $D_i := \{ k\in \{1,2,\dots,n\}: P^{(k)}_i \neq 0}.

         In the most direct sense, a flux matrix is calculated by summing probability 
         flow between some time $t$ and some discrete lag-time later, 
         $t+\tau$. Given the $N*1$ vector $P^{(k)}$ where each $P{(k)}_i$ represents the 
         probability in bin $i$ of simulation $k$ at the first timestep $t$ in 
         building the flux matrix, the transition matrix may then be esimated by 
              $<T_{i,j}> = \frac{<F>_{i,j}}{P_i}$
\end{Verbatim}
               
               \[<T_{i,j}> = \frac{<F_{i,j}>}{\sum_{k=1}^n P^{(k)}_i} \] 

\begin{Verbatim}[commandchars=\\\{\},codes={\catcode`$=3\catcode`^=7\catcode`_=8}]
              >>> If w\_postanalysis\_matrix calculates the flux on a 
                  sub-WE-iteration basis, does it also store probability on a 
                  sub-WE-iteration basis? 
      2) Average the transition matrices.  Given a collection of $N*N$ 
         flux matrices 
              $\{F^{(1)}, F^{(2)}, ..., F^{(n)}\}$
         we first normalize each row $F_i$ by the total probability in bin $i$
         at the first timestep $t$ in building the flux matrix, we first 
         row-normalize each flux matrix as 
              $T^{(k)}_{i,j} = \frac{F^{(k)}_{i,j}}{P_i}$
\end{Verbatim}
               
               \[T^{(k)}_{i,j} = \begin{cases} 
                                   \frac{F^{(k)}_{i,j}}{P_i} & P_i \neq 0 \\
                                   NAN & P_i=0 
                                 \end{cases}
                              \]
\begin{Verbatim}[commandchars=\\\{\},codes={\catcode`$=3\catcode`^=7\catcode`_=8}]
         for k=1,2,...,n.  We may then obtain an estimate of the transition
         matrix $<T>$ as:
\end{Verbatim}
              \[<T_{i,j}> = \frac{\sum_{k\in D_{i,j}} T^{(k)}}{D_{i,j}^\#} \]
              
\begin{Verbatim}[commandchars=\\\{\},codes={\catcode`$=3\catcode`^=7\catcode`_=8}]
         where $D_{i,j} = \{k\in \{1,2,\dots,n\}: T^{(k)}_{i,j} \neq NAN\}$, and $D_{i,j}$
         represents the number of elements in $D$. Intuitively, we include 

         Either formulation may be valid.  Method (1) places more "emphasis" on 
         high-weight transitions, in that large flux values contribute more to 
         the estimate <T> than do low-weight transitions. Method (2) places equal
         emphasis on each transition observation, regardless of the associated
         weight.

    - In the case of a simulation with recycling conditions, construction of a 
      transition matrix is not so straightforward.  Consider a set of n
      simulations indxed by $k=1,2,\dots ,n$. Let $F^{(k)}$ be a raw, $N*N$ 
      flux matrix calculated by observing probability flux between bins during 
      the period betweenn time $t$ and time $t+\tau$. We may first obtain a
      transition matrix for each simulation $k=1,2,\dots ,n$ as follows: 
              $T^{(k)}_{i,j} = \frac{F^{(k)}_{i,j}}{P_i}$
      Let $B^{(k)} \subset \mathbb{N}$, $|B|<N$ be a collection of bin indices 
      corresponding to those bins from which trajectory walkers are recycled 
      during the course of the simulation.  Because trajectory walkers are 
      continually removed from bins $i{\in}B^{(k)}$, it is unlikely to achieve 
      sufficient sampling of transition events out of bins $i{\in}B^{(k)}$ to reasonably 
      estimate any transition rate $T^{(k)}_{i,j}$ where $i\in B^{(k)}$, 
      $j=1,2,\dots ,N$. Thus, such transition rates are not allowed to
      contribute to the estimate.
      An estimate for the transition matrix is then:
              $<T_{i,j}> = \frac{\sum_{k{\in}I_i} T^{(k)}_{i,j}}{I_i^\#}$ 
          where:
              $I_i := \{k{\in}\{1,\dots,n\}: B^{(k)}{\not}{\owns}i\}$
      and $I_i^#$ denotes the number of elements in $I_i$.
    - Apply averaging.  Averaging could, in theory, take place at the
      transition matrix level (i.e., find the mean of all transition matrices),
      or could occur at the level of populations (average the stationary
      distributions calculated from a sequence of transition matrices; more on 
      this below), or for rates.  Since this script will include transition 
      matrices in its output, I think it makes sense to average at the
      transition matrix level.  This is also consistent with 
      w\_postanalysis\_reweight.


\end{Verbatim}
\end{document}
